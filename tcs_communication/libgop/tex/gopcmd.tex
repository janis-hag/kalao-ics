%\documentstyle[12pt,a4wide]{livre}
%\documentstyle[11pt,a4wide,twoside,supertab]{rapport}
%\documentstyle[11pt,a4wide,twoside,longtable,supertab,epsf,fancybox,french,html]{report}

\documentclass[11pt,a4paper]{report}
\usepackage{alltt}
\usepackage{epsfig}
\usepackage{times}
\usepackage{html}
\usepackage{fancybox}
\usepackage{longtable}
\usepackage{supertab}
\usepackage[francais]{babel}
\usepackage[T1]{fontenc}
\usepackage[isolatin]{inputenc}



\pagestyle{headings}
%\setlength{\topmargin}{0pt}
\hoffset -1in
%\voffset -1in
\setlength{\oddsidemargin}{2.5cm}
\setlength{\evensidemargin}{1.5cm}
%\setlength{\headheight}{0pt}
%\setlength{\headsep}{0pt}
%\setlength{\textheight}{24cm}
\setlength{\textwidth}{17cm}
%\setlength{\footheight}{2cm}
%\setlength{\footskip}{2cm}
\setlength{\parindent}{0pt}

%
% nouvelle indentation de la table des matieres
%
\makeatletter
\setlength{\@tempdima}{20mm}
\makeatother
%
\makeatletter
\newcommand{\fctint}[4]{{#1}{\dotfill}&{#2}
			  \def\argiii{#3}%
                          \ifx\argiii\@empty%
                              \else{#3}%
                          \fi%
                          \def\argiv{#4}%
                          \ifx\argiv\@empty\\%
                              \else\\&Exemple: {\tt {#4}}\\%
                          \fi%
                          }
\newcommand{\ftitre}[1]{\\ \bf\underline {#1} \\ \\}
\makeatother
%
%------------------------+
% MACRO POUR LA COMMANDE |
%------------------------+
% encadre la commande
%\newcommand{\titre}[1] {\begin{flushright}
%                           {\Large\bf\fbox{\fbox{#1}}\\[10mm]}
%                        \end{flushright}}
\newcommand{\titre}[1] {\section{#1}}
%
% ecrit la commande
\newcommand{\cmda}[1]    {{\large\bf {#1} \\[4mm]}}
%
% texte de la commande
\newcommand{\textcmd}[1] {{\rm {#1}  \\[8mm] }}
%\newcommand{\textcmd}[1] {{\rm {#1}}}

%
% format de la commande
\newcommand{\fmtcmd}[2] {{\rm Format: \\[2mm]
                          \hspace{2cm} \bf {#1}\hspace{3mm} {#2} \\[8mm]}}
%
% commande complete
%
% utilisation: \commande{COMMANDE}{texte}{parametre}{texte}
%
%\newcommand{\commande}[4]{{\titre{#1}\cmda{#1}\textcmd{#2}\fmtcmd{#1}{#3}
%                      \textcmd{#4} }}
\newcommand{\commande}[4]{{\titre{#1}\textcmd{#2}\fmtcmd{#1}{#3}
                      \textcmd{#4} }}
%--------------------------+
% MACRO POUR LES EXEMPLES  |
%--------------------------+
%
\newcommand{\exemple}[1]  {{Ex: \hspace{10mm} {#1}}}%
%
%---------------------------------+
% MACRO POUR LES TITRES SOULIGNES |
%---------------------------------+
\newcommand{\titreunderl}[1]{{\rm
                             \underline{#1}
                             \\[4mm]}}

%-------------------------+
% MACRO POUR LES KEYWORDS |
%-------------------------+
%
\def\keywordlabel#1{\bf {#1}\hfil\sl}
\def\keyword{\list{}{\leftmargin 4.7truecm
                     \labelwidth 3.3truecm\advance\labelwidth-\labelsep
                     \let\makelabel\keywordlabel}}
\let\endkeyword\endlist
\def\keywordlabelshort#1{\bf {#1}\hfil\sl}
\def\keywordshort{\list{}{\leftmargin 2truecm
                     \labelwidth 2truecm\advance\labelwidth-\labelsep
                     \let\makelabel\keywordlabelshort}}
\let\endkeywordshort\endlist
%--------------------------+
% MACRO POUR LES REMARQUES |
%--------------------------+
%\font\mathi=ammi10 at 12pt
%
% texte de la commande
\newcommand{\remarque}[1] {\vspace{8mm}{\normalsize\rm\underline {Remarque:}\nopagebreak
                             \\[3mm]\it {#1} }}
\newcommand{\remarques}[1] {\vspace{8mm}{\normalsize\rm\underline {Remarques:}\nopagebreak
                             \\[3mm]\it {#1} }}

%-------------------------------------------------+
% LOGO                                            |
%-------------------------------------------------+

\newcommand{\applic}[0]  {{\sl Application}}
\newcommand{\cmd}[0]  {{\sl Commande}}

\input description.var

%-------------------------------------------------+
% SEPARATIONS
%-------------------------------------------------+


%\begin{latexonly}

\def\separation{\vskip.5\baselineskip%
\hrule width\textwidth height0.2mm\vskip-.2mm%
\penalty-5000%
\hrule width\textwidth height0.2mm\vspace{-3mm}%
}

\def\bigseparation{\vskip\baselineskip%
\hrule width\textwidth height0.2mm\vskip1mm%
\penalty10000%
\hrule width\textwidth height0.2mm\vskip-1.4mm%
\penalty-5000%
\hrule width\textwidth height0.2mm\vskip1mm%
\penalty10000%
\hrule width\textwidth height0.2mm\vspace{-3mm}%
}
%\end{latexonly}


\begin{htmlonly}
\newcommand {\separation} {\hline}
\newcommand {\bigseparation} {\hline}
\end{htmlonly}


%-------------------------------------------------+
% Parametres
%-------------------------------------------------+

\newcommand {\docbeginparams} {\separation \subsubsection*{Param�tres:} \rm \begin{list}{}{\leftmargin=5.5cm\labelsep=.5cm\labelwidth=3cm\listparindent=0cm\parsep=0mm}}
\newcommand {\docbeginparam} {\separation \subsubsection*{Param�tre:} \rm \begin{list}{}{\leftmargin=5.5cm\labelsep=.5cm\labelwidth=3cm\listparindent=0cm\parsep=0mm}}

\newcommand {\docendparam}  {\end{list}}
\newcommand {\docendparams} {\end{list}}

\newcommand {\itemparam} [2] {\item[{\bf{#1}\hfill}] {#2}}


%-------------------------------------------------+
% sujet
%-------------------------------------------------+

\newcommand {\sujet}  [1] {\bigseparation \subsection*{#1}\noindent\par}

%-------------------------------------------------+
% subsujet
%-------------------------------------------------+

\newcommand {\subsujet}  [1] {\separation \subsection*{#1}\par}

%-------------------------------------------------+
% exemple
%-------------------------------------------------+

\newcommand {\docexemples}  {\separation \subsubsection*{Exemples:}}

%-------------------------------------------------+
% exemple
%-------------------------------------------------+

\newcommand {\docexemple}  {\separation \subsubsection*{Exemple:}}

%-------------------------------------------------+
% indentation
%-------------------------------------------------+

\newcommand{\docindent}[1]{%
\begin{list}{}{\leftmargin=2cm}\item {#1}\end{list}}%
\newcommand{\docindentbold}[1]{%
\begin{list}{}{\leftmargin=2cm}\item {\bf{#1}}\end{list}}%

%-------------------------------------------------+
% variables importantes
%-------------------------------------------------+

\newcommand {\docbeginvints} {\separation \subsubsection*{Variables devant �tre initialis�es avant l'appel � cette fonction:} \rm \begin{list}{}{\leftmargin=7cm\labelsep=.5cm\labelwidth=4.5cm\listparindent=0cm\parsep=0mm}}
\newcommand {\docbeginvress} {\separation \subsubsection*{Variables mises � jour par cette commande:} \rm \begin{list}{}{\leftmargin=7cm\labelsep=.5cm\labelwidth=4.5cm\listparindent=0cm\parsep=0mm}}
\newcommand {\docbeginvint} {\separation \subsubsection*{Variable devant �tre initialis�e avant l'appel � cette fonction:} \rm \begin{list}{}{\leftmargin=7cm\labelsep=.5cm\labelwidth=4.5cm\listparindent=0cm\parsep=0mm}}
\newcommand {\docbeginvres} {\separation \subsubsection*{Variable mise � jour par cette commande:} \rm \begin{list}{}{\leftmargin=7cm\labelsep=.5cm\labelwidth=4.5cm\listparindent=0cm\parsep=0mm}}


\newcommand {\itemvar} [3] {\item[{\bf{#1}\hfill}] {#2}}
\newcommand {\itemvarup} [1] {\item[{\bf{#1}\hfill}] {Voir plus haut sous "VARIABLES INTERACTIVES PRINCIPALES"}}

\newcommand {\docendvar}  {\end{list}}
\newcommand {\docendvars} {\end{list}}

%-------------------------------------------------+
% remarque
%-------------------------------------------------+

\newcommand {\docremarque} [1] {\separation \subsubsection*{Remarque:} {\begin{list}{}{\setlength{\leftmargin}{2cm}}\item{\rm {#1}}\end{list}}}

%-------------------------------------------------+
% remarques
%-------------------------------------------------+

\newcommand {\docremarques} [1] {\separation \subsubsection*{Remarques:} {\begin{list}{}{\setlength{\leftmargin}{2cm}}\item{\rm {#1}}\end{list}}}

%-------------------------------------------------+
% description breve
%-------------------------------------------------+

%\newcommand {\descriptionbreve}  [1] {\rm{#1}}
\newcommand {\descriptionbreve}  [1] {{\begin{list}{}{\setlength{\leftmargin}{2cm}}\item{\rm {#1}}\end{list}}}

%-------------------------------------------------+
% synopsis
%-------------------------------------------------+

\newcommand{\synopsis} {\separation \subsubsection*{Synopsis:}}

%-------------------------------------------------+
% description
%-------------------------------------------------+

\newcommand{\descriptionlongue}[1]{\separation \subsubsection*{Description:}{\begin{list}{}{\setlength{\leftmargin}{2cm}}\item{\rm {#1}}\end{list}}}

\newcommand{\indentdescription}[1]{%
\begin{list}{}{\leftmargin=2cm}%
		\item {#1}%
\end{list}}%

%-------------------------------------------------+
% Syntaxe
%-------------------------------------------------+

\newcommand {\docbeginsyntaxe}  {\separation \subsubsection*{SYNTAXE:}  \rm \begin{list}{}{\setlength{\leftmargin}{2cm}}}

\newcommand {\docbeginsyntaxes} {\separation \subsubsection*{SYNTAXES:} \rm \begin{list}{}{\setlength{\leftmargin}{2cm}}}

\newcommand {\itemsyntaxe} [1] {\item {#1}}

\newcommand {\docendsyntaxe}  {\end{list}}
\newcommand {\docendsyntaxes} {\end{list}}

%
%--------------------------------+
% Macro pour les  qualificateurs
%--------------------------------+

\def\titrquas{\separation \subsubsection*{QUALIFICATEURS A DISPOSITION:}}
\def\titrqua{\separation \subsubsection*{QUALIFICATEUR A DISPOSITION:}}

\newlength{\largqual}%
\setlength{\largqual}{\textwidth}%
\addtolength{\largqual}{-2cm}%

\newcommand{\qualif}[2]{%
\begin{list}{}{\leftmargin=2cm\topsep=2pt%
		\partopsep=0pt}%
		\penalty-1000\item {\bf {#1}}\penalty10000%
\end{list}%
\begin{list}{}{\leftmargin=3cm\topsep=1pt%
		\partopsep=0pt}%
		\penalty10000\item {    {#2}}%
\end{list}}%

\newcommand{\indentqualif}[1]{%
\begin{list}{}{\leftmargin=3cm\topsep=1pt%
		\partopsep=0pt}%
		\penalty10000\item {    {#1}}%
\end{list}}%


%-------------------------------------------------+
% MACRO POUR LES PARAMETRES DE LA SECTION GLOBALE |
%-------------------------------------------------+
%
\newcommand {\lkw} [3] {{#1} & {#2} & {#3}\\ \hline}
\newcommand {\shortvar} [6] {{#1} & {#2} & {#3} & {#4} & {#5} & {#6}\\ \hline}
\def\titrpar{\rm \large \underline{Variables du bloc utilis�es par cette commande}}
\newcommand{\titrtabpar}{\multicolumn{1}{l}{\large Variables}
                  &\multicolumn{1}{c}{\large Mode}
                  &\multicolumn{1}{l}{\large Description} \\[1mm] \hline}
\newcommand{\titrtabparsh}{\multicolumn{1}{l}{\large Variables}
                  &\multicolumn{1}{c}{\large Mode}
		  &\multicolumn{1}{l}{\large Variables}
                  &\multicolumn{1}{c}{\large Mode}
		  &\multicolumn{1}{l}{\large Variables}
                  &\multicolumn{1}{c}{\large Mode} \\[1mm] \hline}

\makeatletter
\newcommand{\titrevar}{%
	\@startsection%
		{section}%
		{1}%
		{0pt}%
		{10ex plus 1ex}%
		{2ex plus .2ex}%
		{\raggedright\large\bf}}
\makeatother


\newcommand{\begindefine}{\begin{list}{}{\leftmargin=5.5cm\labelsep=.5cm\labelwidth=4.5cm\listparindent=0cm\parsep=0mm\renewcommand\makelabel{\tt}}}
